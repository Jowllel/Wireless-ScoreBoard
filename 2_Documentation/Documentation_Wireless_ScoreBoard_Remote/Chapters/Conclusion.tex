\chapter{Conclusion}
\label{cha:Conclusion}


\textbf{Project Start}

I've noticed that it was quite hard for me to start with the project. There are so much topics to imply, that it can get overwhelming and very emotional. I've also underestimated the time needed to design 3 different PCBs.

\vspace{5mm}
\textbf{Time Plan}

Well, the GANTT chart as you can see didn't help out much for me. Because it's really hard to plan such a projects timing over half a year. There are so many unsolved problems and in the beginning I haven't been able to estimate the duration of all the processes involved. On the other hand my 2-week plans were working great and helped a lot, to notice what was to do.

\vspace{5mm}
\textbf{HW development}

I've learned much in this project part. Mainly this was the optimization of my workflow in Altium Designer. This also leaded to some more nicely solved solutions. Because I used templates and Libraries wherever I could.

I forgot to add silkscreen to the accelerometers footprint. Right now the IC is really hard to center on the pads, because there is no anchor point to position the IC.

And I also forgot to place texts on the silkscreen besides the buttons. So every time I want to reset the MCU I have to try all the buttons on the PCB.

\vspace{5mm}
\textbf{Outlook}

In the next steps of the Wireless ScoreBoard Remote project, I will work on fixing any issues with the Display control unit and completing the firmware. This involves making sure the display unit communicates well with the remote and testing it thoroughly for accuracy. Additionally, I will finalize the firmware by resolving any remaining tasks or bugs, ensuring the remote works smoothly. By focusing on these tasks, I aim to improve the overall performance and usability of the scoreboard.