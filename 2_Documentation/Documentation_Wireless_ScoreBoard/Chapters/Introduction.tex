\chapter{Introduction}
\label{cha:Introduction}


\section{Mindmap}
\label{sec:Mindmap}

\vspace{1cm}

\begin{figure}[H]
	\centering
    \begin{tikzpicture}[mindmap, grow cyclic, every node/.style=concept, concept color=orange!40, 
        level 1/.append style={level distance=5cm,sibling angle=90},
        level 2/.append style={level distance=3cm,sibling angle=45},]

        \node{WSB Remote}
            child [concept color=blue!30] { 
                node {Software}
                child { node {WSB Lib}}
                child { node {HAL}}
                child { node {USB DFU}}
                child { node {Fuel Gauge}}
            }
            child [concept color=green!30] { 
                node {Hardware}
                child { node {Li-Ion Battery}}
                child { node {nRF24}}
                child { node {Accelero-meter}}
                child { node {STM32}}
                child { node {PWR Bootstrap}}
            }
            child [concept color=red!30] { 
                node {New Skills}
                child { node {STM32 SPI}}
	            child { node {Low Power}}
	            child { node {Fusion 360}}
	            child { node {Git}}
            };
    \end{tikzpicture}  

    \vspace{1cm}

	\caption{Project Mindmap}
	\label{fig:Project Mindmap}
\end{figure}


\section{''Pflichtenheft''}
\label{sec:Pflichtenheft}

\subsubsection{Cost}
I've already bought some boards to prototype the remote as well as the ePaper displays, to test them out. Further expenses e.g. the PCB will be paid by me and shouldn't exceed about 50 CHF, as the remote HW isn't that complicated.

\subsubsection{Time}
The majority of the time in the project, I will work at home because it's a rather big project to execute in one semester. The project will approximately take 100h to complete. Also, the more detailed time plan is in chapter [\ref{sec:GANTT}] and sequential 2-week plans in chapter [\ref{sec:2-week plans}].

\subsubsection{Tools}
To realize this project I will mainly use, the SW STM32CubeIDE with HAL, STM32 CubeProgrammer, Draw.IO and Altium Designer. The documentation is written in LaTeX using Overleaf. And I'm planning to order the PCB on JLCPCB and will populate and reflow the PCB at ETHZ, where I'm also allowed to use the measurement equipment for the commissioning and miscellaneous measurements.

\subsubsection{Technical Details}
\begin{table}[H]
    \centering
    \label{tab:Technical Details}
\begin{tabular}{||c || c | c | c | c  || c ||} 
 \hline
 value &  min. & typ. & max. & unit & description \\ [0.5ex] 
 \hline\hline
  supply voltage & & 5 & & V & over USB \\ 
 \hline
\end{tabular}
    \caption{Technical Details}
\end{table}

\newpage